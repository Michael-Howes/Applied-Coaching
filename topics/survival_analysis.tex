\section{Survival Analysis}

Here we will mostly follow the presentation in Section 3.6 of \citep{efron_2022}. We will also include some references to past qualifying exams questions.

\subsection{Survival functions and hazard rates}

Survival analysis is sometimes called ``time-to-event'' analysis. Our data contains variables $T_i \ge 0$ which record the time at each the ``event'' occurred for subject $i$. Historically, $T_i$ was the time at which the $i$th subject died. However, survival analysis can be used in less morbid applications. The historic development influences the terminology of survival analysis. We will often interpret $T_i \ge t$ as subject $i$ ``surviving'' to time $t$ or subject $i$ being ``at risk'' at time $t$.


\subsubsection*{Survival functions}
An event time $T$ can be thought of as a non-negative random variable. Our goal is to infer the distribution of $T$ for different subjects. We can do this inference in terms of the \emph{survival function} of $T$. The survival function of $T$ is the function $S_T:[0,\infty) \to [0,1]$ given by
\[
    S_T(t) = \bbP(T \ge t).    
\]
In the case when $T$ is continuous, the survival function $S_T(t)$ is equal to $1-F_T(t)$ where $F_T$ is the CDF of the distribution of $T$. Note that the distribution of $T$ is completely determined by $S_T$. 

\subsubsection*{Hazard rates}
The distribution of $T$ can also be described by $T$'s \emph{hazard rate} $h_T(t)$. This is the chance of the event occurring at time $t$ conditional on surviving to at least time $t$. For simplicity, we will consider discrete and continuous distributions separately.\footnote{We will not consider the hazard function of a distribution that is a mixture of discrete and continuous.}  In these two cases the hazard rate is defined as follows:
\begin{align}
    h_T(t)&=\frac{\bbP(T = t)}{S_T(t)} \quad \text{when $T$ is discrete}\label{eq:hazard-discrete}\\
    h_T(t)&= \frac{f(t)}{S_T(t)} \quad \text{when $T$ is continuous with density $f(t)$}\label{eq:hazard-continuous}
\end{align}
Note that in the discrete case, $h_T(t)=0$ for all $t \notin \supp T$. For both discrete and continuous distributions, the survival function $S_T(t)$ can be recovered from the hazard rate $h_T(t)$. In particular,
\begin{align}
    S_T(t)&=\prod_{0 \le u < t}(1-h_T(u)) \quad \text{when $T$ is discrete},\label{eq:surv-haz-discrete}\\
    S_T(t)&=\exp\left(-\int_0^t h_T(u)du\right) \quad \text{when $T$ is continuous}.\label{eq:surv-haz-continuous}.
\end{align}
The distribution of $T$ is therefore determined by the hazard rate $h_T(t)$. Our modeling and inference can will be based on the hazard rates $h_T(t)$. Equations \eqref{eq:surv-haz-discrete} and \eqref{eq:surv-haz-continuous} show that $S_T(t)$ is a \emph{decreasing} function of $h_T(t)$. Higher hazard rates correspond to earlier event times (on average).

\subsection{Censoring}

A complication in survival analysis that our data is often \emph{right censored}. This means that for some subjects, we do not observe the survival time $T_i$. For censored subjects, we only know that $T_i \ge C_i$ where $C_i$ is another random time called the censored time. Typically, $C_i$ will be the time at which the study ends measured from when subject $i$ joins the experiment. We will assume our data is in the following form:
\begin{align}
    O_i &= \min\{C_i,T_i\} \in [0,\infty) \nonumber \\
    \delta_i &=I_{\{O_i = T_i\}} \in\{0,1\}, \quad 1 \le i \le N.\label{eq:surv-model1}
\end{align}
The variable $O_i$ is the observed time for subject $i$. It is either equal to the event time $T_i$ or the censored time $C_i$. The variable $\delta_i$ is an indicator with $\delta_i=1$ meaning that for subject $i$ is not censored.

\subsection{Estimation}

Consider data as \eqref{eq:surv-model1} but with the additional assumption that $T_i \simiid T$ and that $(T_i)_{i=1}^N$ are independent of  $(C_i)_{i=1}^N$. One of the main tasks in survival analysis is using the data in \eqref{eq:surv-model1} to estimate the distribution of $T$. 

\subsection*{The Kaplan--Meier estimate}

Let $E = \{O_i : 1 \le i \le N, \delta_i = 1\}$ be the set of observed event times. For each $t \in E$ defined the following,
\begin{itemize}
    \item The risk set at time $t$, $R(t) = \{i:O_i \ge t\}$.
    \item The number of at risk subjects at time $t$, $n(t) = |R(t)|$.
    \item The number of uncensored event times equal to $t$, $y(t) = |\{i : O_i =t, \delta_i=1\}|$. 
\end{itemize}
Under our i.i.d. assumptions, the conditional distribution of $y(t)$ given $n(t)$ is
\[y(t) \mid n(t) \sim \mathrm{Binom}(n(t),h_T(t)) \text{ for } t \in E. \]
For $t \in E$, we can estimate $h_T(t)$ with the MLE,
\[\hat{h}(t) = \frac{y(t)}{n(t)}. \]
From \eqref{eq:surv-haz-discrete}, we get an estimate of the survival function for $T$,
\[\hat{S}(t) = \prod_{u \in E, u < t}(1-\hat{h}_T(u)). \]
This is called the \emph{Kaplan--Meier estimate} of $S_T$. We can estimate the variance of  $\hat{S}_T$ by \emph{Green-wood's formula}
\[\var\left(\hat{S}(t)\right) \approx \hat{S}(t)^2 \sum_{u \in E, u \le t} \frac{y(u)}{n(u)(n(u)-y(u))}.\]

\subsection*{Parametric modelling}

The MLE $y(t)/n(t)$ can have high variance, especially for large value of $t$ where we expect $n(t)$ to be small. We can reduce the variance of $\hat{S}(t)$ by putting modelling assumptions on the hazard function $h(t)$. Since we have binomial data $y(t)\mid n(t)$, it is natural to use a GLM. Specifically, now assume that
\[
    T_i \simiid T, \quad \logit( h_T(t)) = g(t;\theta),
\] 
where $g(t:\theta)$ is a parametric family of functions on $[0,\infty)$. If we assume that $g(t;\theta)$ can be represented at $g(t;\theta) = \Phi(t)^\top \theta$, then we get a binomial GLM. Specifically, for each $t \in E$,
\[ 
    y(t) \mid n(t) \sim \mathrm{Binom}(n(t), p(t)), \quad \logit(p(t)) = \Phi(t)^\top \theta.    
\]
We can fit a GLM to get the MLE $\hat{\theta}$ of $\theta$. This gives the following estimates of $h_T$ and $S_T$,
\begin{align*}
    \hat{h}(t)&=\logit^{-1}(\Phi(t)^\top \hat{\theta}),\\
    \hat{S}(t)&=\prod_{u \in E, u < t}(1-\hat{h}(u)).
\end{align*}
Standard GLM theory gives the limiting distribution of $\hat{\theta}$. You can then use the delta method to get asymptotic standard deviations for $\hat{h}(t)$ and $\hat{S}(t)$. Here is code to fit such a model in \verb|R|,

\begin{minted}{R} 
    glm(Y/n ~ Phi, family = binomial(link = "logit"), weights = n) 
\end{minted}

Both the Kaplan--Meier estimate and the parametric GLM estimate come up in Question 6 on the 2015 qualifying exam. A parametric modelling question also comes up in Question 1 on the 2018 qualifying exam. There a log link is used instead of a logistic link.

\subsection{Comparing different populations}

In the previous section, we considered the problem of estimating $h_T(t)$ for one distribution $T$. A more common problem is comparing the distribution of $T$ for two different populations. Specifically, suppose now that we have two samples,
\begin{align*}
    &\{(O_i^0,\delta_i^0):1 \le i \le N_0\},\\
    &\{(O_i^1,\delta_i^1) : 1 \le i \le N_1\}.
\end{align*}
We assume that each sample is the form \eqref{eq:surv-model1} with the same independence assumptions as before. Specifically, we assume that $C_i^j,T_i^j$ are all independent of each other and for $j=0,1$ and $1 \le i \le N_j$,
\begin{align*}
    T_i^j& \simiid T^j,\\
    O_i^j &= \min\{C_i^j,O_i^j\},\\
    \delta_i^j &= I_{\{O_i^j = T_i^j\}}.
\end{align*}
We wish to test if $T^0 \stackrel{\text{dist}}{=} T^1$. This can be done by fitting separate hazard rates to each sample as above. You can then use the estimated standard errors to see the two hazard rates are substantially different. 

\subsubsection*{The Log-rank test}

The log-rank test is a non-parametric test of the null $T^0 \stackrel{\text{dist}}{=}  T^1$. As before, define the following quantities 

\begin{itemize}
    \item The set of all uncensored event times across the two groups \[
        E = \{O_i^0 : 1 \le i \le N_0, \delta_i^0=1\} \cup \{O_i^1 : 1 \le i \le n_1, \delta_i^1=1\}.
        \]
    \item The number of at risk subjects in each sample at time $t$, 
    \[n_j(t) = |\{1 \le i \le N_j : O_i^j \ge t\}|.\]
    \item The number of uncensored times in each sample equal to $t$, \[y_j(t) = |\{1 \le i \le N_j : O_i^j =t, \delta_i^j=1\}|. \]
\end{itemize}
Under the null hypothesis $T^0  \stackrel{\text{dist}}{=} T^1$, we have for all $t \in E$, 
\[y_j(t) \mid n_j(t) \sim \mathrm{Binom}(n_j(t), h(t)). \]
If we condition on $y_0(t)+y_1(t)$, we can eliminate the unknown parameter $h(t)$. Specifically, if we set $y(t)=y_0(t)+y_1(t)$ and $n(t)=n_0(t)+n_1(t)$, then
\[y_0(t)\mid n_0(t),n_1(t), y(t) \sim \mathrm{Hypergeometris}(n(t), y(t), n_0(t)). \]
We can thus use this conditional distribution to conduct tests of $T^0 \stackrel{\text{dist}}{=} T^1$.  The log-rank test standardizes and combines the values of $y_0(t)$ in the following way,
\begin{equation}\label{eq:log-rank}
    Z := \frac{\displaystyle{\sum_{t \in E} y_0(t) - \bbE[y_0(t) \mid n_0(t),n_1(t),y(t)]}}{\displaystyle{\sqrt{\sum_{t \in E} \var(y_0(t)\mid n_0(t), n_1(t), y(t))}}} \approx \calN(0,1),
\end{equation}
where 
\begin{align*}
    \bbE[y_0(t) \mid n_0(t),n_1(t), y(t)] &= \frac{y(t)n_0(t)}{n(t)} \quad \text{and }\\
    \var(y_0(t)\mid n_0(t), n_1(t), y(t)) &= y(t)\frac{n_0(t)}{n(t)}\frac{n_1(t)}{n(t)}\frac{n_0(t)+n_1(t)-y(t)}{n(t)-1},\\
\end{align*} 
by the hypergeometric result above. 

The method of combining $|E|$ hypergeometric statistics as in \eqref{eq:log-rank} is called a ``Conchran--Mantel--Haenazel test.'' The log-rank test can also be extended to multiple groups using the Fisher--Yates or multivariate hypergeometric distribution. 

\subsection{Proportional hazard models}

Comparing the hazard functions for two groups can be thought of a regression problem with a binary feature. We'd also like to see how continuous feature affect the hazard times, or how multiple features affect the hazard times. \emph{Proportional hazards models} allow us to do exactly this. Assume that for subjects $1 \le i \le N$, we have
\begin{itemize}
    \item Features $X_i \in \reals^p$,
    \item Observed times $O_i = \min\{C_i,T_i\} \ge 0$,
    \item Censor indicators $\delta_i = I_{\{O_i = T_i\}} \in \{0,1\}$.
\end{itemize}
Our inference will be conditional on the features $X_1,\ldots,X_N$. We will assume the following proportional hazards model for $T_i \mid X_i$,
\begin{align}
    T_i\mid X_i \simind h(t;X_i) = h_0(t)\exp(X_i^\top \beta). \label{eq:prop-haz}
\end{align}
Larger values of $X_i^\top \beta$ correspond to higher hazard functions and thus earlier event times. The unknown function $h_0(t)$ is called the baseline hazard rate.
As before we assume that given the features $X_i$, $C_i$ and $T_i$ are independent. 

\subsubsection*{Estimation}

As before, we will conduct inference by conditioning on the risk sets $R(t)$. This will have the affect of eliminating the baseline hazard $h_0(t)$ which is a nuisance parameter. For simplicity, we will assume that there are no ties in the set $E$.

An important consequence of the proportional hazards model \eqref{eq:prop-haz} is the following. For an individual $i$ with $\delta_i=1$, the probability that they are the member of $R(t_i)$ that dies at time $t$ is 
\[\pi_i(\beta\mid R(t_i)) = \frac{\exp(X_i^\top \beta)}{\sum_{k \in R(t_i)} \exp(X_k^\top \beta)}. \]
We then define the \emph{partial likelihood} to be the product of the above factors 
\[L(\beta) = \prod_{i:\delta_i=1} \pi_i(\beta \mid R(t_i)) = \prod_{i:\delta_i = 1}\frac{\exp(X_i^\top \beta)}{\sum_{k \in R(t_i)} \exp(X_k^\top \beta)}, \]
and the corresponding \emph{log-partial likelihood},
\[\ell(\beta) = \log L(\beta) = \sum_{i:\delta_i = 1} X_i^\top \beta -\log \left(\sum_{k \in R(t_i)} \exp(X_k^\top \beta)\right). \]
We then estimate $\beta$ by maximizing $\ell(\beta)$. That is
\[\hat{\beta} = \argmax_{\beta} \ell(\beta). \]
This is tractable since $\ell(\beta)$ is concave in $\beta$. We also have
\begin{align*}
    \nabla_\beta \ell(\beta) &=\sum_{i :\delta_i = 1}\left( X_i - m_i(\beta)\right)\\
    \nabla^2_\beta \ell(\beta) &=\sum_{i:\delta_i=1}V_i(\beta),
\end{align*}
where
\[m_i(\beta) = \sum_{k \in R(t_i)}\pi_k(\beta \mid R(T_i))X_k,\]
and
\[ V_i(\beta) = \sum_{k \in R(t_i)}\pi_k(\beta \mid R(T_i))(X_k - m_i(\beta))(X_k - m_i(\beta))^\top.\]
The \verb|R| package \verb|survival| is a standard tool for fitting proportional hazards models. Here is code that use this package
\begin{minted}{R}
    S <- Surv(O, delta)
    coxph(S ~ X)
\end{minted}
You can also use \verb|glm| style notation,
\begin{minted}{R}
    S <- Surv(O, delta)
    coxph(S ~ X1 + X2 + X3)
\end{minted}
Intercepts are not fitted in proportional hazards models. The unknown baseline hazard makes intercepts unidentifiable.


\subsubsection*{Inference}
A lot of theoretical statistics gives the normal approximation:
\[\hat{\beta} \approx \calN\left(\beta, \hat{\Sigma}\right), \]
where $\hat{\Sigma} = \left(-\nabla^2_\beta \ell(\hat{\beta})\right)^{-1}$. This normal approximation can be used to test if $C^\top\beta = 0$ for matrix $C \in \reals^{p \times k}$. For example, we can test hypothesis of the form $\beta_j=0$ for some feature $j$. You can use the Wald, Score or likelihood tests from 300B.

It turns out that the score test is closely related to the log-rank test above. Suppose that we have a single binary feature $X_i \in \{0,1\}$, and we want to test 
\[T_i \mid X_i=0 \stackrel{\text{dist}}{=} T_i \mid X_i = 1.\] 
If we use the proportional hazards model \eqref{eq:prop-haz} with $X_i$ as the only covariate, then this is equivalent to testing $\beta=0 \in \reals$. The score test statistic is
\[S_{\text{score}} = \frac{\nabla_\beta \ell(0)^2}{-\nabla_\beta^2 \ell(0)}.\]
That is, the score evaluated at $0$ divided by the Fisher information at $0$. It turns out that $S_{\text{score}}$ is exactly equal to $Z^2$ where $Z$ is as in \eqref{eq:log-rank}. 